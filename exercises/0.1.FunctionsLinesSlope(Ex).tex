\begin{exercises} 

\item (modified from NCTM Illuminations) The table below displays data that relate the number of oil changes per year and the
    cost of engine repairs.  To predict the cost of repairs from the number of oil
    changes, use the number of oil changes as the $x$ variable and the engine repair cost
    as the $y$ variable.  
    \begin{center}
        \begin{tabular}[h!]{|c|c|}
            \hline
            Oil Changes Per Year & Cost of Repairs (\$) \\ \hline \hline
            3 & 300 \\ 
            5 & 300 \\ 
            2 & 500 \\
            3 & 400 \\ 
            1 & 700 \\
            4 & 400 \\
            6 & 100 \\
            4 & 250 \\ 
            3 & 450 \\
            2 & 650 \\
            0 & 600 \\ 
            10 & 0 \\
            7 & 150 \\ \hline
        \end{tabular}
    \end{center}

    \ba
    \item Using graph paper make a plot of the data on appropriate axes.
    \item Do the data appear linear?  Why or why not?
    \item Pick two representative points from the data and use them to write the
        equation of a line that {\it fits} the data.  Plot your line on top of your data
        and discuss how well your line fits the
        data.   (This may take a few attempts.)
    \item Despite how well your data fit a linear model, it is not entirely sensible to
        use a linear model for this data.  Why?
    \ea
    
\begin{exerciseSolution}
\end{exerciseSolution}


\item The population of a city, $P$, in millions, is a function of $t$, the number of
    years since 1960, so $P = f(t)$.  Which of the following statements explains the
    meaning of $f(38) = 8$ in terms of the population of this city?
    \ba
        \item The population of this city in the year 38 is 8 million people.
        \item The population of this city in the year 8 is 38 million people.
        \item The population of this city in the year 1968 is 38 million people.
        \item The population of this city in the year 1998 is 8 million people.
    \ea
\begin{exerciseSolution}
    The independent variable is the number of years after 1960 so the ``38'' represents
    the year 1998.  Hence, the phrase ``The population of this city in the year 1998 is 8
    million people'' is the correct phrase.
\end{exerciseSolution}

\item Determine the slope and $y$-intercept of the line whose equation is $-4y + 6x + 8 =
    0$.

\begin{exerciseSolution}
    Solving for $y$ we see that 
    \[ 4y = 6x + 8 \quad \implies \quad y = \frac{6}{4} x + 2. \]
    Therefore the slope is $m=\frac{6}{4} = \frac{3}{2}$ and the $y$-intercept is $2$.
\end{exerciseSolution}

\item The value of a car in 1990 is \$13,100 and the value is expected to go down by \$80
    per year for the next 7 years.  Write a linear function for the value, $V$, of the
    1990 car as a function of the number of years from 1990, $x$.  

\begin{exerciseSolution}
    \[ V(x) = -80x + 13100 \] 
\end{exerciseSolution}
\end{exercises}
\afterexercises
