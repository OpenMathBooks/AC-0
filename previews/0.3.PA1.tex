\begin{pa} \label{PA:0.3}
The goal of this activity is to explore and experiment with the function
\[ F(x) = Af(B(x-C))+D. \]
The values of $A$, $B$, $C$, and $D$ are constants and the function $f(x)$ will be
henceforth called the {\it parent function}.  To facilitate this exploration, use the
applet located at \\
\href{http://www.geogebratube.org/student/m93018}{http://www.geogebratube.org/student/m93018}.
\ba
    \item Let's start with a simple parent function: $f(x) = x^2$.
        \bei
            \item Fix $B=1$, $C=0$, and $D=0$.  Write a sentence or two describing the
                action of $A$ on the function $F(x)$.
            \item Fix $A=1$, $B=1$, and $D=0$.  Write a sentence of two describing the
                action of $C$ on the function $F(x)$.
            \item Fix $A=1$, $B=1$, and $C=0$.  Write a sentence of two describing the
                action of $D$ on the function $F(x)$.
            \item Fix $A=1$, $C=0$, and $D=0$.  Write a sentence of two describing the
                action of $B$ on the function $F(x)$.
        \eei
    \item In part (a) you have made conjectures about what $A$, $B$, $C$, and $D$ do to a
        parent function graphically.  Test your conjectures with the functions $f(x) =
        |x|$ (typed \texttt{abs(x)}), $f(x) = x^3$, $f(x) = \sin(x)$, $f(x) = e^x$ (typed
        \texttt{exp(x)}), and any other function you find interesting. 
\ea
\end{pa} \afterpa
