\begin{activity}\label{A:0.2.3}\cite[p.9]{nonlinear}
    Uncontrolled geometric growth of the bacterium {\it Escherichia coli (E. Coli)} is the
    theme of the following quote taken from the best-selling author Michael Crichton's
    science fiction thriller, {\it The Andromeda Strain:}
    \begin{quote}
        ``The mathematics of uncontrolled growth are frightening.  A single cell of the
        bacterium E. coli would, under ideal circumstances, divide every twenty minutes.
        That is not particularly disturbing until you think about it, but the fact is that
        that bacteria multiply geometrically: one becomes two, two become four, four
        become eight, and so on.  In this way it can be shown that in a single day, one
        cell of E. coli could produce a super-colony equal in size and weight to the
        entire planet Earth.''
    \end{quote}
    \ba
        \item Write an equation for the number of E. coli cells present if a single cell
            of E. coli divides every 20 minutes. 
        \item How many E. coli would there be at the end of 24 hours?
        \item The mass of an E. coli bacterium is $1.7 \times 10^{-12}$ grams, while the
            mass of the Earth is $6.0 \times 10^{27}$ grams.  Is Michael Crichton's claim
            accurate?  Approximate the number of hours we should have allowed for this
            statement to be correct?
    \ea
\end{activity}\aftera
