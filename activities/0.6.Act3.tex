\begin{activity}\label{A:0.6.3}

%The purpose of this activity is to prompt students to start thinking about vertical asymptotes (and division by 0) in terms of limits.  
	
\ba
		\item Suppose $f(x) = x^{2}+ 3x + 2$ and $g(x) = x - 3$.  
			\begin{enumerate}
				\item What is the behavior of the function $h(x) = \displaystyle{\frac {f(x)}{g(x)}}$ near $x = -1$? (i.e. what happens to $h(x)$ as $x\to -1$?) near $x = -2$? near $x = 3$?
				\item What is the behavior of the function $h(x) = \displaystyle{\frac {g(x)}{f(x)}}$ near $x = -1$? near $x = -2$? near $x = 3$?
			\end{enumerate}
		\item Suppose $f(x) = x^{2} - 9$ and $g(x) = x - 3$.  
			\begin{enumerate}
				\item What is the behavior of the function $h(x) = \displaystyle{\frac {f(x)}{g(x)}}$ near $x = -3$? (i.e. what happens to $h(x)$ as $x\to -3$?) near $x = 3$?
				\item What is the behavior of the function $h(x) = \displaystyle{\frac {g(x)}{f(x)}}$ near $x = -3$? near $x = 3$?
			\end{enumerate}
		\item Suppose $f(x) = \sin{x}$ and $g(x) = x$.
			\begin{enumerate}
				\item What is $f(0)$? What is $g(0)$?
				\item What is the behavior of the function $h(x) = \displaystyle{\frac {f(x)}{g(x)}}$ near $x = 0$? (i.e. what happens to $h(x)$ as $x\to 0$?)
				\item What is the behavior of the function $h(x) = \displaystyle{\frac {g(x)}{f(x)}}$ near $x = 0$?
			\end{enumerate}
\ea

\end{activity}\aftera
