\begin{activity}\label{A:0.6.3}
% The purpose of this activity is to prompt students to start thinking about vertical asymptotes (and division by 0) in terms of limits.  
\ba
		\item Suppose $f(x) = x^{2}+ 3x + 2$ and $g(x) = x - 3$.  
			\begin{enumerate}
                \item[(i)] What is the behavior of the function $h(x) = \displaystyle{\frac {f(x)}{g(x)}}$ near $x = -1$? (i.e. what happens to $h(x)$ as $x\to -1$?) near $x = -2$? near $x = 3$?
                \item[(ii)] What is the behavior of the function $k(x) = \displaystyle{\frac {g(x)}{f(x)}}$ near $x = -1$? near $x = -2$? near $x = 3$?
			\end{enumerate}
		\item Suppose $f(x) = x^{2} - 9$ and $g(x) = x - 3$.  
			\begin{enumerate}
                \item[(i)] What is the behavior of the function $h(x) = \displaystyle{\frac {f(x)}{g(x)}}$ near $x = -3$? (i.e. what happens to $h(x)$ as $x\to -3$?) near $x = 3$?
                \item[(ii)] What is the behavior of the function $k(x) = \displaystyle{\frac {g(x)}{f(x)}}$ near $x = -3$? near $x = 3$?
			\end{enumerate}
% 		\item Suppose $f(x) = \sin{x}$ and $g(x) = x$.
% 			\begin{enumerate}
%                 \item[(i)] What is $f(0)$? What is $g(0)$?
%                 \item[(ii)] What is the behavior of the function $h(x) = \displaystyle{\frac {f(x)}{g(x)}}$ near $x = 0$? (i.e. what happens to $h(x)$ as $x\to 0$?)
%                 \item[(iii)] What is the behavior of the function $k(x) = \displaystyle{\frac {g(x)}{f(x)}}$ near $x = 0$?
% 			\end{enumerate}
\ea

\end{activity}
\begin{smallhint}
    Watch carefully for division by zero.  That is where a vertical asymptote is possible.
\end{smallhint}
\begin{bighint}
    Watch carefully for division by zero.  That is where a vertical asymptote is possible.
\end{bighint}
\begin{activitySolution}
   \ba
        \item For $f(x) = x^2+3x+2$ and $g(x) = x-3$ when we form the function $h(x) =
            f(x)/g(x)$ we get $h(x) = \frac{x^2+3x+2}{x-3}$.  Notice that $h(-1) = 0$ and
            $h(-2) = 0$ since $f(-1) = f(-2) = 0$ and $g(-2) \neq 0$.  At $x=3$, on the
            other hand, $g(x) = 0$ and there is a vertical asymptote on the function
            $h(x)$.

            For $k(x) = g(x) / f(x)$.  The vertical asymptotes occur at $x=-2$ and $x=-1$
            and a zero occurs at $x=3$.
        \item For $f(x) = x^2-9$ and $g(x) = x-3$ we see immediately that $f(x)$ can be
            factored to $f(x) = (x-3)(x+3)$ and hence the function $h(x) = f(x)/g(x)$
            simplifies to $h(x) = x+3$ when $x$ is not $3$.  When $x=3$ there is a
            discontinuity in the graph but this is not a vertical asymptote since the
            discontinuity can be removed algebraically.

            For $k(x) = g(x)/f(x)$ we simplify to $k(x) = 1/(x+3)$ when $x$ is not $3$.
            Therefore there is a vertical asymptote at $x=-3$ and a removable
            discontinuity at $x=3$.
   \ea
\end{activitySolution}

\aftera
