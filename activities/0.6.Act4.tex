\begin{activity}\label{A:0.6.4}
%The purpose of this activity is to encourage students to think of horizontal asymptotes in terms of "dominance" of the numerator over denominator (or vice versa).  
\ba
		\item Suppose $f(x) = x^{3} + 2x^{2}-x + 7$ and $g(x) = x^{2} + 4x + 2$.  
			\begin{enumerate}
                \item[(i)] Which function dominates as $x \to \infty$?
				\item[(ii)] What is the behavior of the function $h(x) = \displaystyle{\frac {f(x)}{g(x)}}$ as $x \to \infty$?
				\item[(iii)] What is the behavior of the function $k(x) = \displaystyle{\frac {g(x)}{f(x)}}$ as $x \to \infty$?
			\end{enumerate}
		\item Suppose $f(x) = 2x^{4} - 5x^{3} + 8x^{2} - 3x - 1$ and $g(x) = 3x^{4} - 2x^{2} + 1$
			\begin{enumerate}
                \item[(i)] Which function dominates as $x \to \infty$?
				\item[(ii)] What is the behavior of the function $h(x) = \displaystyle{\frac {f(x)}{g(x)}}$ as $x \to \infty$?
				\item[(iii)] What is the behavior of the function $k(x) = \displaystyle{\frac {g(x)}{f(x)}}$ as $x \to \infty$?
			\end{enumerate}
        \item Suppose $f(x) = e^{x}$ and $g(x) = x^{10}$.
			\begin{enumerate}
                \item[(i)] Which function dominates as $x \to \infty$ as $x \to \infty$?
				\item[(ii)] What is the behavior of the function $h(x) = \displaystyle{\frac {f(x)}{g(x)}}$ as $x \to \infty$?
				\item[(iii)] What is the behavior of the function $k(x) = \displaystyle{\frac {g(x)}{f(x)}}$ as $x \to \infty$?
			\end{enumerate}
\ea

\end{activity}
\begin{smallhint}
    The term ``dominate'' means that as $x$ gets large the function that ``dominates'' is
    much larger.
\end{smallhint}
\begin{bighint}
    The term ``dominate'' means that as $x$ gets large the function that ``dominates'' is
    much larger.
\end{bighint}
\begin{activitySolution}
   \ba
        \item When considering the functions $f(x) = x^3 + 2x^2 - x + 7$ and $g(x) =
            x^2+4x+2$, the cubic function, $f(x)$, will dominate as $x \to \infty$.
            Therefore, as $x \to \infty$, $h(x) = f(x)/g(x)$ will go to infinity and $k(x)
            = g(x)/f(x)$ will go to zero.
        \item When considering the functions $f(x) = 2x^{4} - 5x^{3} + 8x^{2} - 3x - 1$
            and $g(x) = 3x^{4} - 2x^{2} + 1$ we see that the polynomials are the same
            degree hence one does not dominate over the other.  As $x\to\infty$ we see
            that $h(x) \to \frac{2}{3}$ and $k(x) \to \frac{3}{2}$.
        \item When considering the functions $f(x) = e^x$ and $g(x) = x^{10}$ we know that
            the exponential function will dominate over the power function based on our
            work in prior activities.  Therefore, $h(x) = f(x)/g(x) \to \infty$ as
            $x\to\infty$ and $k(x) = g(x)/f(x) \to 0$ as $x \to \infty$.

   \ea
\end{activitySolution}

\aftera
