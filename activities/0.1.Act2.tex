\begin{activity}\label{A:0.1.2}
Find the equation of the line with the given information.
\ba
\item The line goes through the points $(-2,5)$ and $(10,-1)$.
\item The slope of the line is $3/5$ and it goes through the point $(2,3)$.
\item The $y$-intercept of the line is $(0,-1)$ and the slope is $-2/3$.
\ea

\end{activity}
\begin{smallhint}
   \ba
        \item Recall that $m = \frac{\Delta y}{\Delta x}$ and use the point slope form of
            the line.
        \item You are given a slope and a point.  Which form of the line should you use?
        \item You are given a slope and the $y$ intercept.  Which form of the line should
            you use?
   \ea
\end{smallhint}
\begin{bighint}
   \ba
        \item The slope is $m = \frac{\Delta y}{\Delta x} = \frac{5-(-1)}{(-2)-10} =
            \frac{6}{-12} = -\frac{1}{2}$.  Now use the point-slope form of the line.
        \item The point-slope form of the line is best here.
        \item The slope-intercept form of the line is best here.
   \ea
\end{bighint}
\begin{activitySolution}
   \ba
        \item The slope is $m = \frac{\Delta y}{\Delta x} = \frac{5-(-1)}{(-2)-10} =
            \frac{6}{-12} = -\frac{1}{2}$.  Hence,
            \[ y - 5 = -\frac{1}{2} \left( x-(-2) \right) \]
            which can be rewritten as
            \[ y = -\frac{1}{2} \left( x+2 \right) + 5. \]
        \item Using the point-slope form of the line we get
            \[ y - 3 = \frac{3}{5} \left( x-2 \right) \]
            which can be rearranged to 
            \[ y = \frac{3}{5} \left( x-2 \right) + 3. \]
        \item Using the slope-intercept form of the line we get
            \[ y = -\frac{2}{3} x - 1. \]
   \ea
\end{activitySolution}
\aftera
