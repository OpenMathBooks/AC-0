\begin{activity}\label{A:0.6.4}

%The purpose of this activity is to encourage students to think of horizontal asymptotes in terms of "dominance" of the numerator over denominator (or vice versa).  
	
\ba
		\item Suppose $f(x) = x^{3} + 2x^{2}-x + 7$ and $g(x) = x^{2} + 4x + 2$.  
			\begin{enumerate}
				\item Which function dominates as $x \to \infty$?
				\item What is the behavior of the function $h(x) = \displaystyle{\frac {f(x)}{g(x)}}$ as $x \to \infty$?
				\item What is the behavior of the function $h(x) = \displaystyle{\frac {g(x)}{f(x)}}$ as $x \to \infty$?
			\end{enumerate}
		\item Suppose $f(x) = 2x^{4} - 5x^{3} + 8x^{2} - 3x - 1$ and $g(x) = 3x^{4} - 2x^{2} + 1$
			\begin{enumerate}
				\item Which function dominates as $x \to \infty$?
				\item What is the behavior of the function $h(x) = \displaystyle{\frac {f(x)}{g(x)}}$ as $x \to \infty$?
				\item What is the behavior of the function $h(x) = \displaystyle{\frac {g(x)}{f(x)}}$ as $x \to \infty$?
			\end{enumerate}
        \item Suppose $f(x) = e^{x}$ and $g(x) = x^{10}$.
			\begin{enumerate}
				\item Which function dominates as $x \to \infty$ as $x \to \infty$?
				\item What is the behavior of the function $h(x) = \displaystyle{\frac {f(x)}{g(x)}}$ as $x \to \infty$?
				\item What is the behavior of the function $h(x) = \displaystyle{\frac {g(x)}{f(x)}}$ as $x \to \infty$?
			\end{enumerate}
\ea

\end{activity}\aftera
